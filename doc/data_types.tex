\section{Data Types} \label{sec:data_type}
\noindent
The same physical quantity is expressed by different notations, units, or even data formats by integrating multiple models, especially the third-party tools.
For instance, temperature in Celsius, Fahrenheit, or Kelvin in data types of $|int|$, $|unsigned|$, $|float|$, $|double|$.
KitFox defines the data types of shared physical quantities used across multiple physical models.

{
\fontsize{10pt}{11pt}\selectfont
\begin{alltt}
typedef int Index;            // KITFOX_DATA_INDEX
typedef uint64_t Comp_ID;     // Pseudo Component ID
typedef uint64_t Count;       // KITFOX_DATA_COUNT
typedef double Meter;         // KITFOX_DATA_LENGTH
typedef double MeterSquare;   // KITFOX_DATA_AREA
typedef double MeterCube;     // KITFOX_DATA_VOLUME
typedef double Second;        // KITFOX_DATA_TIME, KITFOX_DATA_PERIOD
typedef double Hertz;         // KITFOX_DATA_CLOCK_FREQUENCY
typedef double Volt;          // KITFOX_DATA_VOLTAGE, KITFOX_DATA_THRESHOLD_VOLTAGE
typedef double Joule;         // KITFOX_DATA_ENERGY
typedef double Watt;          // KITFOX_DATA_POWER, KITFOX_DATA_TDP
typedef double Kelvin;        // KITFOX_DATA_TEMPERATURE, KITFOX_DATA_AMBIENT_TEMPERATURE
typedef double Unitless;      // KITFOX_DATA_FAILURE_RATE
\end{alltt}
}

\subsection{Dimension} \label{subsec:dimension}
\noindent
Dimension data ($|KITFOX\_DATA\_DIMENSION|$) are used to define the geometry of the component, e.g., floorplans. 
Dimension uses the left-bottom corner for the orientation.
Note that the $|area|$ can be calculated by a physical model without known width and height information, so $|area|$ is not necessarily the same as $|width|\times|height|$.
If the $|area|$ is non-zero, $|get\_area|$ function returns $|area|$, or otherwise this function returns $|width|\times|height|$.

{
\fontsize{10pt}{11pt}\selectfont
\begin{alltt}
class dimension_t \{
public:
    dimension_t();

    Index die_index; // Index of the processor die in 3D package (0 by default and 2D)
    char die_name[MAX_KITFOX_COMP_NAME_LENGTH]; // Name of the processor die
    Meter left, bottom; // Left and bottom orientation of the component
    Meter width, height; // x, y-dimensional length
    MeterSquare area; // Area
    
    const MeterSquare get_area(); // Returns area (if non-zero) or width*height
    const Meter get_center_x() const; // Returns left+width/2
    const Meter get_center_y() const; // Returns bottom+height/2
    const Meter get_left() const; // Returns left
    const Meter get_right() const; // Returns left+width
    const Meter get_bottom() const; // Returns bottom
    const Meter get_top() const; // Returns bottom+height
    void clear(); // Reset function
\};
\end{alltt}
}

\subsection{Grid} \label{subsec:grid}
\noindent
Thermal grid ($|KITFOX_DATA_THERMAL_GRID|$) stores 3D mesh data of thermal grid cells. Note that this data cannot be transmitted across MPI processes.

{
\fontsize{10pt}{11pt}\selectfont
\begin{alltt}
template <typename T>
class grid_t \{
public:
    grid_t();
    grid_t(Meter CellWidth, Meter CellHeight,\(\backslash\)
           unsigned Columns, unsigned Rows, unsigned Dies);
    
    unsigned cols(); // Returns x
    unsigned rows(); // Returns y
    unsigned dies(); // Returns z
    void operator=(grid_t<T> &g); // Assignment
    T& operator[](unsigned i) // Indexing operator
    T& operator()(unsigned Columns, unsigned Rows, unsigned Dies);
    void reserve(unsigned Columns, unsigned Rows, unsigned Dies);
    void clear(); // Reset function
    Meter grid_cell_width, grid_cell_height; // Grid cell geometry
private:
    unsigned x, y, z; // x, y, z dimensional size
\};
\end{alltt}
}

\subsection{Counter} \label{subsec:counter}
\noindent
Access counters ($|KITFOX_DATA_COUNTER|$) represent microarchitectural activities of a component.
The dynamic power is calculated based on the counter numbers.
$|switching|$ counts represent logical switching activities of the component such as combinational logic, functional unit, wire, etc. 
$|read|$ and $|write|$ counts mean typical and read/write behaviors of storage units including cache, memory, array, buffer, queue, etc. 
For caches, $|read|$ and $|write|$ counts implicitly include the accesses to tag arrays, so tag access counts must not be counted in addition read/write counters.
$|read\_tag|$ and $|write\_tag|$ are used to count the read/write accesses to tag arrays. 
Note that both read and write cache misses are counted by $|read\_tag|$, and $|write\_tag|$ is not the counter for write misses but used to represent tag write/update behaviors such as the tag update of reservation station. 
In fact, a cache miss incurs multiple increments of counters. 
First, it reads the tag array ($|read\_tag++|$) and finds a miss. 
MSHR is a separate unit than the main cache array, so the MSHR has to be modeled with another pseudo component. 
The MSHR counters must not be mixed with the cache counters. 
This is the same for other scheduling buffers in the cache including prefetch and writeback buffers. 
When the cache miss is served from the next-level cache or memory, the cache array is updated ($|write++|$). 
For a fully-associative cache, a cache access requires the search of multiple entries to determine hit or miss, which is different from the access behavior or typical set-associative cache. 
To account for this case, $|search|$ counter is used to represent the entry-searching behavior of the cache array. 
In this situation, $|read|$, $|write|$, or tag accesses indicate the direct access to a cache entry, not including the search behavior.

{
\fontsize{10pt}{11pt}\selectfont
\begin{alltt}
class counter_t \{
public:
    counter_t();
    
    Count switching; // Logical switching activities (e.g., logics, wires, ALUs, etc.)
    Count read; // Read accesses of a storage unit (e.g., caches, arrays, queues, etc.)
    Count write; // Write accesses of a storage unit
    Count search; // Tag search of a CAM structure (fully-associative cache).
    Count read_tag; // Tag read accesses (e.g., cache miss)
    Count write_tag; // Tag write accesses (e.g., tag update)
    /* Additional counters for DRAM */
    Count precharge; // Precharge counts
    Count background_open_page_high; 
    Count background_open_page_low; 
    Count background_closed_page_high; 
    Count background_closed_page_low;
    
    void operator=(const counter_t &c);
    const counter_t operator+(const counter_t &c);
    const counter_t operator-(const counter_t &c);
    const counter_t operator*(const Count &c);
    const counter_t operator/(const Count &c);
    void clear(); // Reset function
\};
\end{alltt}
}

\subsection{Unit Energy} \label{subsec:unit_energy}
\noindent
Unit energy ($|KITFOX_DATA_UNIT_ENERGY|$) defines the per-access energies of distinct access behaviors defined in the $|counter_t|$ class.

{
\fontsize{10pt}{11pt}\selectfont
\begin{alltt}
class unit_energy_t \{
public:
    unit_energy_t();
    
    Joule baseline; // Activity-independent dynamic energy consumption per clock cycle
    Joule switching;
    Joule read;
    Joule write;
    Joule read_tag;
    Joule write_tag;
    Joule search;
    Joule precharge;
    Joule background_open_page_high; 
    Joule background_open_page_low; 
    Joule background_closed_page_high; 
    Joule background_closed_page_low;
    Joule leakage; // Leakage energy dissipation per clock cycle
    
    void operator=(const unit_energy_t &e);
    const unit_energy_t operator+(const unit_energy_t &e);
    const unit_energy_t operator-(const unit_energy_t &e);
    void clear(); // Reset function    
\};
\end{alltt}
}

\subsection{Energy} \label{subsec:energy}
\noindent
Energy class ($|KITFPX_DATA_ENERGY|$) represents the generic energy definition comprised of dynamic and leakage portions.

{\fontsize{10pt}{11pt}\selectfont
\begin{alltt}
class energy_t \{
public:
    energy_t();
    
    Joule total;
    Joule dynamic;
    Joule leakage;
    
    Joule get_total(); // Returns total (if non-zero) or dynamic+leakage
    void operator(const energy_t &e);
    const energy_t operator+(const energy_t &e);
    const energy_t operator-(const energy_t &e);
    void clear(); // Reset function
\};
\end{alltt}}

\subsection{Power} \label{subsec:power}
\noindent
Power ($|KITFOX_DATA_POWER|$ or $|KITFOX_DATA_TDP|$) represent generic power (or TDP) definition comprised of dynamic and leakage portions.

{
\fontsize{10pt}{11pt}\selectfont
\begin{alltt}
class power_t \{
public:
    power_t();
    
    Watt total;
    Watt dynamic;
    Watt leakage;
    
    Watt get_total(); // Returns total (if non-zero) or dynamic+leakage
    void operator(const power_t &e);
    const power_t operator+(const power_t &e);
    const power_t operator-(const power_t &e);
    void clear(); // Reset function
\};
\end{alltt}
}

\subsection{Data Conversion} \label{subsec:data_conversion}
\noindent
Several operators are defined in $|ei-defs.h|$ to simplify the common data conversions, such as energy from/to power.

{
\fontsize{10pt}{11pt}\selectfont
\begin{alltt}
inline power_t operator*(const energy_t &e, const Hertz &f); // Power=Energy*Freq
inline power_t operator/(const energy_t &e, const Second &t); // Power=Energy/Time
inline energy_t operator/(const power_t &p, const Hertz &f); // Energy=Power/Freq
inline energy_t operator*(const power_t &p, const Second &t); // Energy=Power*Time
\end{alltt}
}

