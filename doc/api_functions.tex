\section{List of API Functions} \label{sec:list_of_api_functions}
\noindent
This section lists the API functions of KitFox.
For the examples of using these API functions, refer to the later sections of this document.

\subsection{KitFox Constructor and Configuration Functions} \label{subsec:kitfox_configuration_functions}
\noindent
The KitFox constructor needs one input argument for serial simulation and two arguments for parallel simulation.
$|KitFoxComm|$ is an MPI intra-communicator between KitFox processes in parallel simulations, and $|InterComm|$ is an MPI inter-communicator between KitFox servers and clients (i.e., user simulator components).
Detailed implementation about the MPI communicators will be explained in the later sections of this document (Section \ref{sec:examples}).
{
\fontsize{10pt}{11pt}\selectfont
\begin{alltt}
kitfox_t(MPI_Comm *KitFoxComm = NULL, MPI_Comm *InterComm = NULL);
\end{alltt}
}

\noindent
The following function configures the pseudo component hierarchy, based on the input configuration file and initializes physical models.
{
\fontsize{10pt}{11pt}\selectfont
\begin{alltt}
void configure(const char *ConfigFile);
\end{alltt}
}

\noindent
This function finds the pseudo component with $|ComponentName|$ and returns the integer ID assigned by KitFox.
{
\fontsize{10pt}{11pt}\selectfont
\begin{alltt}
const Comp_ID get_component_id(std::string ComponentName);
\end{alltt}
}

\noindent
This function finds the pseudo component with $|ComponentID|$ and returns the string name (i.e., component path in the input configuration).
{
\fontsize{10pt}{11pt}\selectfont
\begin{alltt}
std::string get_component_name(Comp_ID ComponentID);
\end{alltt}
}

\subsection{Computational Functions} \label{subsec:computational_functions}
\noindent
This function triggers the power calculation of the energy library model associated with a pseudo component, based on the access counters. Calculated power is stored in the queue of the pseudo component.
{
\fontsize{10pt}{11pt}\selectfont
\begin{alltt}
void calculate_power(Comp_ID ComponentID, Second Time, Second Period, counter_t Counter);
\end{alltt}
}

\noindent
This function triggers the temperature calculation of the pseudo component. 
Before calculating temperature, this function internally synchronizes the power data across pseudo components.
It recursively aggregates the power numbers from the leaf components up to the pseudo component that is associated with the thermal library.
After power data synchronization, the pseudo components that are designated as the floorplans of the thermal library store the up-to-date power data.
Then, the temperature calculation function of the thermal library is invoked.
After calculating the thermal field, the floorplan components are updated with the new temperature values stored in the data queues.
When inserting the thermal data, it also synchronizes the pseudo components by recursively inserting the temperature value of the floorplan to the data queue of every descendent component.
This assumes that no placement information is necessarily provided with the descendant components of the floorplan, and hence uniform thermal field is assumed within the floorplan.
If descendent components are associated with energy library models, inserting new temperature data invokes the library-update functions that handle leakage-feedback interactions.
{
\fontsize{10pt}{11pt}\selectfont
\begin{alltt}
void calculate_temperature(Comp_ID ComponentID, Second Time, Second Period);
\end{alltt}
}

\noindent
This function triggers the calculation of failure rate $\lambda t$. 
Calculated failure rates are stored in the data queue of pseudo components. 
Temperature, voltage, and clock frequency data of the pseudo component calculating the failure rate must be up to date before calling this function. 
Otherwise, the simulation terminates by detecting the synchronization error.
{
\fontsize{10pt}{11pt}\selectfont
\begin{alltt}
void calculate_failure_rate(Comp_ID ComponentID, Second Time, Second Period);
\end{alltt}
}

\subsection{Data Queue Functions} \label{subsec:data_queue_functions}
\noindent
This function inserts the data into the queue of pseudo components. 
Return value is an error\_code that must be equal to $|KITFOX_QUEUE_ERROR_NONE|$ if the operation is successful.
{
\fontsize{10pt}{11pt}\selectfont
\begin{alltt}
template <typename T>
const int push_data(Comp_ID ComponentID, Second Time, Second Period, int DataType, T *Data);

template <typename T>
const int push_data(Comp_ID ComponentID, Second Time, int DataType, T *Data);
\end{alltt}
}

\noindent
This function retrieves the data from the queue of pseudo components. 
Return value is also an error\_code that must be equal to $|KITFOX_QUEUE_ERROR_NONE|$ if the operation is successful.
{
\fontsize{10pt}{11pt}\selectfont
\begin{alltt}
template <typename T>
const int pull_data(Comp_ID ComponentID, Second Time, Second Period, int DataType, T *Data);

template <typename T>
const int pull_data(Comp_ID ComponentID, Second Time, int DataType, T *Data);
\end{alltt}
}

\noindent
This function overwrites the data in the queue of pseudo component. 
Return value is an error\_code that must be equal to $|KITFOX_QUEUE_ERROR_NONE|$ if the operation is successful.
{
\fontsize{10pt}{11pt}\selectfont
\begin{alltt}
template <typename T>
const int overwrite_data(Comp_ID ComponentID, Second Time, Second Period, int DataType, T *Data);

template <typename T>
const int overwrite_data(Comp_ID ComponentID, Second Time, int DataType, T *Data);
\end{alltt}
}

\noindent
These functions synchronize the queue data of pseudo components. 
An error\_code is returned, which has to be $|KITFOX_QUEUE_ERROR_NONE|$ if the operation is successful.
{
\fontsize{10pt}{11pt}\selectfont
\begin{alltt}
const in synchronize_data(Comp_ID ComponentID, Second Time, Second Period, int DataType);

template <typename T>
const int synchronize_and_pull_data(Comp_ID ComponentID, Second Time, Second Period,\(\backslash\)
                                    int DataType, T *Data);
                                    
template <typename T>
const int push_and_synchronize_data(Comp_ID ComponentID, Second Time, Second Period,\(\backslash\)
                                    int DataType, T *Data);
                                    
template <typename T>
const int overwrite_and_synchronize_data(Comp_ID ComponentID, Second Time, Second Period,\(\backslash\)
                                         int DataType, T *Data);
\end{alltt}
}

