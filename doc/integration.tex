\section{Integration of Models (for Advanced Users)}
\noindent
This section is intended for advanced users who plan to integrate new models into KitFox framework. 

\subsection{Library}
\noindent
All models integrated in KitFox are categorized into classes called \emph{library}. 
$|model\_library\_t|$ is the base class for all physical models, which is defined as follows. 
The constructor take the pointer to the pseudo component and model library type ($|KITFOX_LIBRARY_ENERGY_MODEL|$, $|KITFOX_LIBRARY_THERMAL_MODEL|$, or $|KITFOX_LIBRARY_RELIABILITY_MODEL|$). 
$|update_library_variable|$ is the runtime variable update function that is registered as a callback function of the data queue (refer to Section \ref{sec:data_queues}).
{
\fontsize{10pt}{11pt}\selectfont
\begin{alltt}
class model_library_t \{
public:
    model_library_t(pseudo_component_t *PseudoComponent, int Type);
    virtual \(\sim\)model_library_t();
    
    int type; // Library type
    
    /* Callback function associated with the data queue to update runtime variables 
    and states of subclass model */
    virtual bool update_library_variable(int Type, void *Value, 
                                         bool IsLibraryVariable = false) = 0;
                                         
    /* Initialization function to be defined by subclass model */
    virtual void initialize() = 0;
    
protected:
    pseudo_component_t *pseudo_component; // Pointer to pseudo component
\};
\end{alltt}
}

\noindent
There are three different derivative classes defined; \emph{energy}, \emph{thermal}, \emph{reliability} libraries.

\subsubsection{Energy Library}
\noindent
Energy library is one of the subclasses of $|model_library_t|$ and the base class for energy/power models. 
Possible models ($|Type|$) are $|KITFOX_LIBRARY_DRAMSIM|$, $|KITFOX_LIBRARY_DSENT|$, $|KITFOX_LIBRARY_INTSIM|$, and $|KITFOX_LIBRARY_MCPAT|$. 
$|get_unit_energy|$ function returns the $|unit_energy|$ class that includes per-access energies of different access types (refer to Section \ref{sec:data_type}). 
$|get_runtime_power|$ is called inside $|calculate_power|$ function of KitFox (refer to Section \ref{sec:calculation_of_physical_properties}) and returns the runtime power including leakage. 
If the model support area estimation, $|get_area|$ can be used, otherwise a trivial value is returned. 
$|get_tdp_power|$ calculates and returns the maximum power based on the assumption of peak access counts and temperature.
{
\fontsize{10pt}{11pt}\selectfont
\begin{alltt}
class energy_library_t : public model_library_t \{
public:
    energy_library_t(pseudo_component_t *PseudoComponent, int Type);
    virtual \(\sim\)energy_library_t();

    int type; // Model type
    
    /* Returns per-access energies of different access types. */
    virtual unit_energy_t get_unit_energy() = 0;
    
    /* Calculates TDP power w.r.t (user defined) max duty cycle 
    and peak temperature. */
    virtual power_t get_tdp_power(Kelvin MaxTemperature = 373.0) = 0;
    
    /* Calculates runtime power with respect to counters. */
    virtual power_t get_runtime_power(Second Time, Second Period, counter_t Counter) = 0;
    
    /* Calculates area (if possible) */
    virtual MeterSquare get_area(void) = 0;
\};
\end{alltt}
}

\subsubsection{Thermal Library}
Thermal library is one of the subclasses of $|model_library_t|$ and the base class for thermal models. 
Possible models ($|Type|$) are $|KITFOX_LIBRARY_3DICE|$, $|KITFOX_LIBRARY_HOTSPOT|$, and $|KITFOX_LIBRARY_MICROFLUIDICS|$. 
$|push_floorplan_power|$ function maps the power input to the floorplan power structure of the thermal model, and $|calculate_temperature|$ can only be called after $|push_floorplan_power|$ is called for all floorplans to update the power inputs. 
This sequence is implemented in the $|calculate_temperature|$ of KitFox, so the thermal library only has to implement the details of power mapping and temperature calculation. 
Note that $|calculate_temperature|$ does not have a return value. 
After this function is called, KitFox calls $|get_thermal_grid|$, $|get_floorplan_temperature|$, or $|get_point_temperature|$ to update the data queues of pseudo components, so $|calculate_temperature|$ only has to update the internal thermal states and be ready for subsequent data queries. 
Other floorplan mapping functions provide useful information about the floorplans.

{
\fontsize{10pt}{11pt}\selectfont
\begin{alltt}
class thermal_library_t : public model_library_t \{
public:
    thermal_library_t(pseudo_component_t *PseudoComponent, int Type);
    virtual \(\sim\)thermal_library_t();
    
    int Type; // Model type

    /* Calculates transient or steady-state temperature. */
    virtual void calculate_temperature(Second Time, Second Period) = 0;
    
    /* Returns 3D thermal grid stack of source layers. */
    virtual grid_t<Kelvin> get_thermal_grid(void) = 0;
    
    /* Returns representative floorplan temperature. */
    virtual Kelvin get_floorplan_temperature(Comp_ID ComponentID, 
                   int Type = KITFOX_TEMPERATURE_MAPPING_UNKNOWN) = 0;
    
    /* Returns point temperature on the die. */
    virtual Kelvin get_point_temperature(Meter X, Meter Y, Index Layer) = 0;
    
    /* Maps the power to the internal floor-plan structure or array. */
    virtual void push_floorplan_power(Comp_ID ComponentID, power_t PartitionPower) = 0;

    /* Other floorplan mapping functions */
    void add_pseudo_floorplan(std::string ComponentName, Comp_ID ComponentID);
    const Index get_pseudo_floorplan_counts(void);
    const Comp_ID get_pseudo_floorplan_id(std::string ComponentName);
    const char* get_pseudo_floorplan_name(Comp_ID ComponentID);
    const Comp_ID get_pseudo_floorplan_id(unsigned i);
\};
\end{alltt}
}

\subsubsection{Reliability Library}
\noindent
Thermal library is one of the subclasses of $|model_library_t|$ and the base class for reliability models. 
Possible model ($|Type|$) is $|KITFOX_LIBRARY_FAILURE|$. 
$|get_failure_rate|$ calculates the instantaneous failure rate $\lambda$ for given temperature, voltage, frequency, and duration (i.e., period). 
Cumulative calculation of the failure rate (refer to Section \ref{sec:calculation_of_physical_properties}) is handled in KitFox, and this function only needs to calculate the failure rate without considering the failure rate history.
{
\fontsize{10pt}{11pt}\selectfont
\begin{alltt}
class reliability_library_t : public model_library_t \{
public:
    reliability_library_t(pseudo_component_t *PseudoComponent, int Type);
    virtual \(\sim\)reliability_library_t();
    
    int Type; // Model type
    
    virtual Unitless get_failure_rate(Second Time, Second Period, Kelvin Temperature,\(\backslash\)
    Volt Vdd, Hertz Frequency) = 0;
\};
\end{alltt}
}

\subsection{Wrapper Class of Integrated Models}
\noindent
For each model being integrated into the library, a wrapper class is created to define the details of virtual functions of base classes. 
For instance, HotSpot is a thermal model and takes $|thermal_library_t|$ as the base class that again takes the base class $|model_library_t|$. 
The following shows an example of HotSpot wrapper class.
{
\fontsize{10pt}{11pt}\selectfont
\begin{alltt}
#include "library.h"

class thermallib_hotspot : public thermal_library_t \{
    thermallib_hotspot(pseudo_component_t *PseudoComponent);
    \(\sim\)thermallib_hotspot();
    
    virtual void initialize() \{ 
        /* Define the initialization of HotSpot variables */ 
    \}
    
    virtual void calculate_temperature(Second Time, Second Period) \{
        /* Define the temperature calculation method of HotSpot */
    \}
    
    virtual grid_t<Kelvin> get_thermal_grid() \{
        /* Returns the 3D (or 2D) thermal grid for only source layers */
    \}
    
    virtual Kelvin get_point_temperature(Meter X, Meter Y, Index Layer) \{
        /* Returns the temperature of the given point on the die */
    \}
    
    virtual void push_floorplan_temperature(Comp_ID ComponentID, power_t PartitionPower) \{
        /* Map the power input to the power structure of floor-plans. If the thermal model 
        uses string name to identify the floor-plan, the base class provides
        get\_pseudo\_floorplan\_name to retrieve the floor-plan name based on ComponentID.
        This name-ID mapping is created in KitFox, so the thermal library can readily use it.*/
        std::string floorplan_name = get_floorplan_name(ComponentID);
    \}
    
    virtual void update_library_variable(int Type, void *Value, bool IsLibraryVariable) \{
        /* This function implements the update of variable. Note that this function is called 
        when there is data insertion into the queue of pseudo components. Type denotes
        data type. If data type is not the KitFox-defined data such as KITFOX\_DATA\_TEMPERATURE,
        then IsLibraryVariable must be false.
        */
    \}
private:
    /* Other HotSpot variables or structures */
\};
\end{alltt}
}

\subsubsection{Parsing Input Configuration}
\noindent
With integrated input configuration, the wrapper class must be aware of parsing the input to initialize the variable of the model in the $|initialize|$ function. 
The $|pseudo_component_t|$ provides methods to read parameters from the input configuration file. 
The following is the list of $|libconfig|$-related functions that $|pseudo_component_t|$ provides.
{
\fontsize{10pt}{11pt}\selectfont
\begin{alltt}
libconfig::Setting& get_setting() const;
libconfig::Setting& get_library_setting() const;
const bool exists_library(void);
const bool exists(std::string &Var, bool InPathLookup = false);
const bool exists(const char *Var, bool InPathLookup = false);
const bool exists_in_library(std::string &Var, bool InPathLookup = false);
const bool exists_in_library(const char *Var, bool InPathLookup = false);
libconfig::Setting& lookup(std::string &Var, bool InPathLookup = false);
libconfig::Setting& lookup(const char *Var, bool InPathLookup = false);
libconfig::Setting& lookup_in_library(std::string &Var, bool InPathLookup = false);
libconfig::Setting& lookup_in_library(const char *Var, bool InPathLookup = false);
\end{alltt}
}

\noindent
The following shows an example of retrieving input parameters via $|pseudo_component_t|$. 
Note that the base class $|model_library_t|$ has the pointer to the pseudo component.
{
\fontsize{10pt}{11pt}\selectfont
\begin{alltt}
void thermalllib_hotspot::initialize() \{
    /* This function looks up "grid_rows" parameters only in the library of
    this pseudo component. If not defined, it throws an exception error. */
    unsigned grid_rows = pseudo_component->lookup_in_library("grid_rows");
    
    /* This function does recursive lookup of parameter "voltage" in the library of
    this pseudo component and all of its ancestor components. */
    unsigned voltage = pseudo_component->lookup_in_library("voltage", true);
    
    /* It can be tested if a parameter is defined in the input configuration to do 
    conditional initialization of a variable. */
    double clock_frequency;
    if(pseudo_component->exists_in_library("clock_frequency")) \{
        clock_frequency = pseudo_component->lookup_in_library("clock_frequency");
    \}
    else \{
        clock_frequency = 1e9;
    \}
    
    /* libconfig setting can be returned as well, which provides useful functionality
    for arrays, groups, etc. Refer to libconfig documentation. */
    Setting &libconfig_setting = pseudo_component->lookup_in_library("layers");
    for(unsigned i =0; i < libconfig_setting.getLength(); i++) \{
        /* Initialize the parameters of each layer of the package. */
    \}
\}
\end{alltt}
}

